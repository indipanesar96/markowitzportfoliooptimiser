\documentclass{article}
\usepackage[utf8]{inputenc}
\usepackage{geometry}
\usepackage{graphicx}
\usepackage{float}
\geometry{legalpaper, portrait, margin=1.5in}
\usepackage{amsmath}
\title{Computational Finance with C++}
\author{CID: 01805027}
\date{10 June 2020}
\usepackage[thinc]{esdiff}
\usepackage{hyperref}
\begin{document}
\renewcommand*{\arraystretch}{1.5}

\maketitle
\section{Introduction} 
\label{sec:introduction}

Since their inception, economists, mathematicians, physicists and computer scientists the like have sought to 'beat' the markets using developments in their respective fields.


maybe capm doesnt need to be mentioned? capm states investors and mean varianceefficient optimisers, and so efficient frontier must've come first

A long-standing theory of the markets is the Capital Asset Pricing Model (CAPM) which states that a market participant may be rewarded with returns above the market rate proportionate to the amount of risk taken on. This gave rise 



MENTION CAPM BUTCAPM SAYS YOURE REWARDED PROPORTIONAL TO HOW MUCH NON-DIVERSIFIABLE RISK YOU TAKE ON, oN THE FRONTIER GRAPH WE LOOK AT ALL THE VARIANCE ASSOCIATED WITH THE STOCK (DIVERSIFIABLE AND NOT)



\subsection{Theory}
\label{sec:theory}
REMEMBER TO CITE THE NOTES

As aforementioned, we employ Lagrange Multipliers to constrain the portfolio problem, allowing us to adhere to real-world constraints whilst also seeking to minimise the variance of the portfolio.

In Equation (\ref{portfolio_variance}), we define the variance of the Markowitz portfolio, $\sigma_{p}^{2}$, with a preceding factor of $\frac{1}{2}$ for convenience. $w_i$ is the weight of asset $i$ in the portfolio and $\sigma_{ij}$ is the covariance between stocks $i$ and $j$.

\begin{equation}
\dfrac{1}{2} \sigma^{2}_{p} =  \dfrac{1}{2} \sum_{i,j=1}^{n} w_{i} \sigma_{ij} w_{j}
\label{portfolio_variance}
\end{equation}

The variance of the portfolio is minimised via the Lagrange method while ensuring the following two constraints are also upheld. 

\begin{equation}
\sum_{i=1}^{n} w_{i} \overline{r_i} - \overline{r_p} = 0
\label{target_return_constraint}
\end{equation}

Above, we mandate that the portfolio return is equal to the value $r_p$. By setting $r_p$, we may effectively 'choose' the portfolio return we desire and thus the weights the optimisation method yields.

Below, is simple the constraint that the sum of the portfolio weights must be equal to unity. This ensures that we do not over or under-allocate the funds available to us.

\begin{equation} 
\sum_{i=1}^{n} w_{i} - 1 = 0
\label{weights_constraint}
\end{equation}

\begin{equation} 
L(w, \lambda, \mu) =  \dfrac{1}{2} \sum_{i,j=1}^{n} w_{i} \sigma_{ij} w_{j} 
	-\lambda \left( \sum_{i=1}^{n} w_{i} \overline{r_i} - \overline{r_p}) \right)
	- \mu \left( \sum_{i=1}^{n} w_{i} - 1 \right)
\label{lagrangian}
\end{equation}

Finally, Equation(\ref{lagrangian}) shows the Lagrangian we seek to minimise. 

For convenience when scaling to a large number of assets and to code in the possibility of introducing further constraints, we introduce the following matrix notation:

\begin{description}
\item [$\bullet$ $\emph{w} = (w_1,..., w_n) \in \Re^n$ ] - a vector of $n$ weights for $n$ assets
\item [$\bullet$ $\overline{\emph{r}} = (\overline{r_1},..., \overline{r_n}) \in \Re^n$ ] - a vector of $n$ expected asset returns
\item [$\bullet$ $\emph{1} = (1,..., 1) \in \Re^n$ ] - the unit vector $n$
\item [$\bullet$ $\emph{0} = (0,..., 0) \in \Re^n$ ] - the zero vector of length $n$
\end{description}

The Lagrangian in matrix form is shown below in Equation (\ref{vector_lagrangrian}) where $\Sigma \in \Re^{n \times n}$ is the covariance matrix of the assets, explicitly; $\Sigma_{ij} = \sigma_{ij}$ as defined above.

\begin{equation} 
L(\textbf{w}, \lambda, \mu)  = \dfrac{1}{2} \textbf{w}'\Sigma\textbf{w}
-\lambda \left( \textbf{w}'\overline{\textbf{r}} - \overline{r_p}\right)
-\mu \left( \textbf{w}'\textbf{e} - 1\right)
\label{vector_lagrangrian}
\end{equation}

Now, in order to minimise the Lagrangian it must be differentiated with respect to the weights, {\textbf{w}}. Allowing us to find the rate of change of the system's Lagrangian with respect to the portfolio weights themselves and setting the resulting equation to zero will give us the turning point of the function.

\begin{equation}
\diff{L(\textbf{w}, \lambda, \mu)}{\textbf{w}'} =  \Sigma\textbf{w}
-\lambda \overline{\textbf{r}}
-\mu\textbf{e} = 0
\label{optimality_lagrangian}
\end{equation}

It may be seen by inspection that the second derivative of the Lagrangian with respect to $\textbf{w}$  is indeed positive for all values of $\textbf{w}$ meaning that this is indeed a minimum and that the function itself is concave.

With Equations (\ref{target_return_constraint}) and (\ref{weights_constraint}) also required for optimality, we may write the system of $n+2$ equations as a single matrix equation written in the form $Ax = b$, below in Equation (\ref{matrix_eqn}). $A$ is an $n+2$ square matrix and the two column vectors both have dimensions $(n+2) \times 1$ respectively. Later we seek to solve this numerically via the Quadratic Conjugate Method to find the vector of weights $w$ that yield the desired portfolio return, $r_p$, the solution to Equation (\ref{matrix_solved}).

\begin{equation}
\begin{bmatrix}
\Sigma & -\overline{\textbf{r}} & -\textbf{e} \\
-\overline{\textbf{r}}'  & 0 & 0 \\
-\textbf{e}' & 0 & 0 
\end{bmatrix}
\begin{bmatrix}
\textbf{w}\\
\lambda \\
\mu
\end{bmatrix}
=
\begin{bmatrix}
\textbf{0}\\
-\overline{r_p}\\
-1
\end{bmatrix}
\label{matrix_eqn}
\end{equation}


\begin{equation}
\begin{bmatrix}
\textbf{w}\\
\lambda \\
\mu
\end{bmatrix}
=
\begin{bmatrix}
\Sigma & -\overline{\textbf{r}} & -\textbf{e} \\
-\overline{\textbf{r}}'  & 0 & 0 \\
-\textbf{e}' & 0 & 0 
\end{bmatrix}^{-1}
\begin{bmatrix}
\textbf{0}\\
-\overline{r_p}\\
-1
\end{bmatrix}
\label{matrix_solved}
\end{equation}


\section{Implementation}
\label{sec:implementation}

\subsection{Code}
\label{sec:implementation_code}


With a distinct focus on code readability, maintainability and efficiency, the backtesting frameworks, Quadratic Conjugate Method and other necessary tools to backtest the Markowitz optimised portfolio were written in C++ (which may be seen explicitly in the  \hyperref[sec:code]{code subsection} of the \hyperref[sec:appendix]{Appendix}).


The project code was split into five distinct project directories for the easy of maintenance and cleanliness. The directories and their contents are outlined briefly for context in the following sections.

\subsubsection{Utility}
\label{sec:utility}

The Utility contains the classes and structs \textit{VectorUtil, Matrix, RunConfig} and \textit{Results}. 
The first two housed small building-block-esque functions upon which the entirety of the application is based. As such, it was vital that these methods used to perform arithmetic operations on the rank 1 and 2 tensors were well tested and efficient. Creating these classes first greatly increased the speed and accuracy of development of the more complex classes that followed.

\textit{RunConfig}  and \textit{Results} are both C++ structs used to group pieces of information in a convenient manor. The former contained parameters that were pertinent to a particular run of the backtest. It was useful to run the application on small and medium sized subsets of the data for debugging purposes and so with each of these came a set of parameters (in/out of sample window lengths, for example) that were specific to the dataset in question. It became cleaner and more convenient to group these parameters together into one object (a \textit{RunConfig} struct) and pass this to \textit{Portfolio::backtest()}.  
The same philosophy was behind the use of the \textit{Results} struct as a return type from the aforementioned portfolio method as a way to group multiple result outputs together, instead of passing each object by reference or pointer individually. 

I found this line of reasoning to be especially pertinent with method signatures since their readability contributes massively to the readability of the application as a whole.  


TALK ABOUT GENERIC TYPES USED IN VECTORUTIL

\subsubsection{Repository}
\label{sec:repository}

Repository houses the classes and methods used to load data in from the comma separated value files. The functionality here was mostly provided with few other editions required.


\subsubsection{Estimator}
\label{sec:parameter_estimation}

This directory holds the \textit{ParamaterEstimator} class and its header, a prime example for the use case of a singleton object design pattern. The class holds public static methods that calculate various parameters such as the means, covariances and standard deviations of the aforementioned vector and self-made \textit{Matrix} objects passed to it. It maintains no state and has no attributes; testament to the functional approach to programming commonly used when encoding mathematical function calculation.

\subsubsection{Optimiser}
\label{sec:portfolio_optimiser}

The \textit{PortfolioOptimiser} class and header are held in this directory and are where the Quadratic Conjugate Method is implemented. It was necessary for this class to maintain a state, having attributes on it that would be required across multiple of its methods. These attributes consisted of constant parameters from the \textit{RunConfig} of the current backtest, such as the initial values for the Lagrange Multipliers $\lambda$ and $\mu$ and the tolerance to which we declare the optimisation method as having converged.

In addition, a getter and setter are employed to set the target return for the optimiser, $\overline{r_p}$ as the same instance of the \textit{PortfolioOptimiser} is used to solve for all different target returns.

\subsubsection{Backtest}
\label{sec:backtestcode}

The Backtest directory is where the \textit{Portfolio} class, header and \textit{main.cpp} were stored. The \textit{main.cpp} acted as the entry into the application; where the \textit{RunConfig}s were initialised and sent to the aforementioned \textit{Portfolio::backtest()} method.

TALK MORE ABOUT THIS

TALK THROUGH THE FLOW OF THE CODE FROM START TO QUADRATIC CONJUGATE METHOD


\subsection{Quadratic Conjugate Method}
\label{sec:qcm}


FLOW CHART ALGO DIAGRAM

\section{Results \& Discussion}
\label{sec:results}

BASICALLY TABLE THAT ALREADY GETS PRINTED OUT

EFFICIENT FRONTIER PLOT

TIME TAKEN TO RUN THE SCRIPT

EXPLAIN WHY STANDARD DEVS GETHIGHER AS RETURNS GET HIGHER - BECASUE WE CANT SEE IT IN THE DATA 9SINCE EXCESSIVELY LARGE RETURNS ARE LESS COMMON) -> MORE NOISE AND ERATICISM IN WEIGHTS PRODUCTION 

METHOD TENDS TO A LIMIT WHERE IT CANT PROCURE MORE RETURN SIMPLY BECAUE PORT RETURN IS WEIGHTS . RETURNS SO PORT RETURN IS LIMITED TO AT BEST, THE SAME RETURN AS THE BEST ASSET SEEN IN THE MARKET. (NO SHORT SELLING) - MAKE UP A SCENARIO WHERE WE SEE IT SHORT SELL A STOCK WITH A NEGATIVE RETURN --> METHOD WORKS


\section{Conclusions}
\label{sec:concs}


DECENT WAY TO OPTIMISE GIVEN HOW FAST CONVERGENCE IS




\begin{thebibliography}{}
\label{sec:thebibliography}
	\bibitem{pwatson} Watson, P. (2005). Ideas: A History of Thought and Invention from Fire to Freud. New York: HarperCollins Publishers.
	
	
\end{thebibliography} 



\section{Appendix} 
\label{sec:appendix}

subsection{Code} 
\label{sec:code}


\end{document}